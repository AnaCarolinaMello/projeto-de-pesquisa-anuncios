\documentclass{beamer}

\usetheme{Copenhagen}

\usepackage[brazilian]{babel}
\usepackage[utf8]{inputenc}
\usepackage{amsmath}
\usepackage{amsfonts}
\usepackage{amssymb}
\usepackage{mathtools}
\usepackage{graphicx}
\usepackage{hyperref}
\usepackage{physics}

\title{Projeto de Pesquisa}
\author{Ana Carolina Caldas de Mello e João Pedro Queiroz Rocha}
\date{Abril de 2024}

\begin{document}

\maketitle

\begin{frame}
    \frametitle{Contexto do Problema}
    Atualmente, no mercado de publicidade onlines, os publishers enfrentam desafios constantes relacionados ao desempenho, à monetização e à performance de seus sites. Um dos grandes fatores para se alcançar esses objetivos é a eficiência dos algoritmos de inserção de anúncio, que garante mais eficácia na exibição de publicidades e, consequentemente, melhoria nas métricas de análise e performace do google.
\end{frame}

\begin{frame}
    \frametitle{Questão-Problema}
    Apesar de existirem diversos algoritmos de inserção de anúncio, nem todos consegue mitigar todos os problemas que o mercado proporciona e todos estão sujeitos a erros, falhas ou, até mesmo, poucas funcionalidades. 
\end{frame}

\begin{frame}
    \frametitle{Objetivo Geral}
    Otimizar um algoritmo para a inserção de anúncios visando aumentar a receita e melhorar o desempenho dos sites dos publihsers.
\end{frame}

\begin{frame}
    \frametitle{Objetivos Específicos}
    \begin{enumerate}
       \item Analisar métricas de performace dos diferentes algoritmos e formas de inserção de anuncios.
       \item Produzir 5 modelos baseados nos pontos fortes dos códigos analisados.
       \item Investigar qual dos algoritmos criados possui a melhor performance.
     \end{enumerate}
\end{frame}

\begin{frame}
    \frametitle{Justificativa}
    Um algoritmo de inserção de anúncios eficiente é fundamental para necessidades crescentes dos publishers em relação à melhoria nas métricas de performance e análise do google. Ademais, otimização da inserção do ads, pode melhorar o carregamento da página, assim aumentando a retenção e engajamento do usuário.
\end{frame}

\end{document}